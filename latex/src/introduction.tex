Quantum computing has become a hot topic in cryptographic research in the last few years. In particular, thanks to the Shor's algorithm \cite{ShorAlgo}, it promises to disrupt well assessed cyrptographic standards, as the RSA cryptosystem and DLOG based cryptosystems. Although the threats of Quantum Computing have been well known for a while, the recent development of physical systems to build fueled the ineterest in developing Quantum resistant protocols, with particualr focus on asymmetric cryptosystem, which are predominantely threatened by the rise of Quantum Computing. The urgence of this matter lead even the NIST to call for a Post Quantum Cryptography Stndardization \cite{NIST}.\\
This interest spawned a variety of Post Quantum Proposals based on known Quantum resistant primitives. In this work the main focus is on Lattice Based primitives, which have proven to be some of the most promising primitives for building Post Quantum asymmetric cryptosystems. In the recent years the research in this field focused on the Ring Lerning With Error problem on Ideal Lattices, which allowed to create cheap and practical cryptosystems due to its versatility. Nevertheless, the lack of a comprehensive security proof tracing the security of the problem to Hard Problems on General Lattices raised a few eyebrows. For this very reason, some researchers have been trying to push for a reevaluation of problems on Generic Lattices, showing their practicality as a basis for Post Quantum cryptosystem.\\
The goal of this work is to examine advantages and disadvantages of Ideal and Generic Lattices, with a particular focus on the Learning With Error and the Ring Learning With Error, with the ultimate goal to present a practical cryptosystem in the context of Generic Lattices. The candidate for this protocol was found in Frodo \cite{frodo}, a Key Exchange protocol presented by Bos, Costello et al. After a due theoretical introduction and a presentation of Frodo and a competitor protocol, New Hope \cite{newhope}, the focus of this work will move on presenting a versatile and ready to use implementation of the former, which aims to improve on the original implementation of Frodo both on performance and ease of use.