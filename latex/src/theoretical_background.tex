\section{Lattices}

The starting point for lattice based cryptography is, quite evidently, lattices. The first part of this chapter focuses on giving an adequate background on lattices. For the sake of brevity and clarity, the following sections will focus on lattices on $\mathbb{R}^n$. Although lattices can potentially be defined using any vector space, this would overburden this introductory chapter with unnecessary notation, distracting the reader from the concepts necessary for building lattice based cryptographic primitives.

\subsection{Generic Lattices}\label{sec:bg:g_lattice}

\begin{definition}[Lattice]
Let $\{b_0,\ldots,b_{n-1}\}$ be a set of linearly independent vectors in $\mathbb{R}^m$, with $m\geq n$. The lattice generated by $\{b_0,\ldots,b_{n-1}\}$ is defined as the set of integer linear combinations of the $b_i$
\begin{equation*}
\mathscr{L}=\left\{\sum_{i=0}^{n-1}l_ib_i | l_i\in\mathbb{Z}\right\}
\end{equation*}
The set $B=\{b_0,\ldots,b_{n-1}\}$ is called a basis of the lattice, $n$ is called lattice rank, and $m$ lattice dimension. If $m=n$, the lattice is said to be a full rank lattice.
\end{definition}

\begin{definition}[Span of a Lattice]
Let $\mathscr{L}$ be a lattice and $B$ a basis for $\mathscr{L}$. The span of the lattice is defined as the linear space spanned by its vectors:
\begin{equation*}
span(\mathscr{L}) = span(B) = \{B\mathbf{x}|\mathbf{x}\in\mathbb{R}^n\}
\end{equation*} 
\end{definition}

\begin{remark}
It is immediate that given two lattices $\mathscr{L}$, $\mathscr{L}'$, then $\mathscr{L}\subset \mathscr{L}' \Rightarrow span(\mathscr{L}) \subset span(\mathscr{L}')$.
\end{remark}

It is worth to introduce an alternative notation for lattices. Let's consider the matrix $B$, with columns $b_0,\ldots,b_{n-1}$. It is easy to see that the lattice generated by $b_0,\ldots,b_{n-1}$ is the set $\mathscr{L}=\{B\mathbf{v}|\mathbf{v}\in\mathbb{Z}^n\}$.\\
Moreover, all of the lattices interesting for lattice based cryptography are full rank lattices, so from now on whenever a lattice is mentioned, it will be assumed to be full rank, i.e. $\mathscr{L}\subset\mathbb{R}^n$.\\
Let's introduce some useful notions on lattices.

\begin{definition}[Minimum distance]
The minimum distance of a lattice $\mathscr{L}$ is defined as the minimum distance between any two lattice points
\begin{equation*}
\lambda = min\{\norm{\mathbf{x}-\mathbf{y}}|\mathbf{x}\ne \mathbf{y}, \mathbf{x},\mathbf{y}\in\mathscr{L}\}
\end{equation*}
Since a lattice is an addictive subgroup of $\mathbb{R}^n$, the above definition is equivalent to
\begin{equation*}
min\{\norm{\mathbf{x}}|\mathbf{x}\ne0, \mathbf{x}\in\mathscr{L}\}
\end{equation*}
\end{definition}

The previous definition can be generalized to a set of linearly independent vectors as follows
\begin{definition}[i-th Successive Minimum]
The i-th successive minimum of a lattice is defined as 
\begin{equation*}
\lambda_i = min\{r \in \mathbb{R}|\exists \{\mathbf{v_i}\}_{i=0}^{n-1} \subset \mathscr{L},\text{ }linearly\text{ }independent,\text{ }s.t.\text{ }\norm{\mathbf{v_i}}\leq r\}
\end{equation*}
\end{definition}

\begin{remark}
It is immediate to see that $\lambda=\lambda_0$
\end{remark}

\begin{definition}[Dual Lattice]
The dual lattice $\mathscr{L}^*$ of a lattice $\mathscr{L}$ is defined as the set
\begin{equation*}
\mathscr{L}^*=\{\mathbf{w}\in\mathbb{R}^n|\forall\mathbf{x}\in\mathscr{L}\langle\mathbf{w},\mathbf{x}\rangle\in\mathbb{Z}\}
\end{equation*}
\end{definition} 

\begin{theorem}
Let $\mathscr{L}$ be a lattice, and $B$ its basis, then the matrix $D=B(B^{t}B)^{-1}$ is a basis for $\mathscr{L}^*$
\end{theorem}
\begin{proof}
Let's call $\tilde{\mathscr{L}}$ the lattice generated by $D$.\\
First of all, $\forall{D\mathbf{y}} \in \tilde{\mathscr{L}}$, it holds
\begin{equation*}
D\mathbf{y} = (B(B^{t}B)^{-1})\mathbf{y} = B(B^{t}B)^{-1}\mathbf{y}) \in span(\mathscr{L})
\end{equation*}
moreover, $\forall B\mathbf{x}\in\mathscr{L}$,
\begin{equation*}
\begin{aligned}
\langle D\mathbf{y},B\mathbf{x} \rangle = (D\mathbf{y})^{t}B\mathbf{x} = \mathbf{y}^{t} (B^tB)^{-1}B^tB\mathbf{x}\\
= \mathbf{y}^{t} B^{-1}(B^t)^{-1}B^tB\mathbf{x} = \mathbf{y}^t\mathbf{x} \in \mathbb{Z}^n
\end{aligned}
\end{equation*}
since $\mathbf{x}$ and $\mathbf{y}$ are vectors of coefficients for elements of the lattices $\mathscr{L}$ and $\tilde{\mathscr{L}}$. Then $D\mathbf{y}\in\mathscr{L}^*$, which means that $\tilde{\mathscr{L}}\subset\mathscr{L}^*$.\\
Now, consider $\mathbf{v}\in\mathscr{L}^*$. By definition of dual, $B^t\mathbf{v}\in\mathbb{Z}^n$. As stated above, it also holds that $\mathbf{v}\in span(\mathscr{L})$, which means that $\exists\mathbf{w}\in\mathbb{R}^n$ $s.t.$ $\mathbf{v}=B\mathbf{w}$. Then
\begin{equation*}
\mathbf{v} = B\mathbf{w} = B(B^tB)^{-1}B^tB\mathbf{w} = DB^t\mathbf{v} \in \mathscr{L}^*
\end{equation*}
since $B^t\mathbf{v}\in\mathbb{Z}^n$, which means that $\mathscr{L}^*\subset\tilde{\mathscr{L}}$, thus concluding the proof.
\end{proof}

\begin{definition}[Fundamental Parallelepiped]
Given a lattice $\mathscr{L}$, and $B$ a basis for $\mathscr{L}$, the fundamental parallelepiped of $\mathscr{L}$ is defined as the set
\begin{equation*}
\mathscr{P} = \{B\mathbf{x}|\mathbf{x}\in\mathbb{R}^n, \forall i \in \{0,\ldots,n-1\}:0\leq x_i \leq 1 \}
\end{equation*}
\end{definition}

\begin{remark}
The fundamental parallelepiped is a really useful concept in lattices. Indeed, it is easy to see that placing a fundamental parallelepiped at every vector of the lattice $\mathscr{L}$, it is possible to tessellate the entirety of $span(\mathscr{L})$.
\end{remark}

\begin{definition}[Voronoi Cell]
Given a lattice $\mathscr{L}$, and $B$ a basis for $\mathscr{L}$, the Voronoi cell of $\mathscr{L}$, with center in $\mathbf{y}\in\mathscr{L}$ is defined as the set
\begin{equation*}
\mathscr{V}_\mathbf{y} = \{\mathbf{x}\in\mathbb{R}^n|\forall \mathbf{y}' \in \mathscr{L}\backslash\mathbf{y}, \norm{\mathbf{x}-\mathbf{y}} < \norm{\mathbf{x}-\mathbf{y}'} \}
\end{equation*}
The fundamental Voronoi cell is defined as $\mathscr{V}_0$. 
\end{definition}

\begin{remark}
The geometry of the Voronoi cell depends on the metric used over $\mathbb{R}^n$. For the purpose of lattice based cryptography, this metric will always be the euclidean distance. So, from now on, any time a Voronoi cell is mentioned, it will be intended as the Voronoi cell computed using the euclidean distance.
\end{remark}

The concept of Voronoi cell is especially important for lattice based cryptography. The nature of this importance will be discussed in more detail later.\\
All of these definitions apply to generic lattices, but there are some subclasses of lattices that are of particular interest for lattice based cryptography. Let's explore some of them, which will become relevant in the applications.

\begin{definition}
Let $\mathscr{L}\subset\mathbb{Z}^n$ be a lattice. $\mathscr{L}$ is said $q$-ary for some $q\in\mathbb{Z}^n$, if $q\mathbb{Z}\subset\mathscr{L}$.
\end{definition} 

\begin{remark}
If $\mathscr{L}$ is a $q$-ary lattices, and $\mathbf{x}\in\mathscr{L}$ a vector, then $x+q\mathbb{Z}^n\in\mathscr{L}$, since $q\mathbb{Z}\subset\mathscr{L}$ and $\mathscr{L}$ is an additive group, but this means that $\forall\mathbf{x}\in\mathbb{Z}^n, \mathbf{x}\in\mathscr{L}\Leftrightarrow\mathbf{x}$ $mod$ $q\in\mathscr{L}$.
\end{remark}

This family of lattices is quite important in applications, where there is the need to deal with finite precision, making lattices over $\mathbb{Z}^n$ quite difficult to manage. With $q$-ary lattices it is possible to reduce the problem of lattice membership of $\mathbf{x}\in\mathbb{Z}^n$ to the much easier and compact problem of lattice membership of $\mathbf{x}$ $mod$ $q$.\\
The other families of lattices fall in some way or another under the polyhedric family that is the one of Ideal Lattices, which really deserves its own section for the importance it has in applications.

\subsection{Ideal Lattices}\label{sec:bg:ideal}

Following the same spirit of the premise made at the beginning of the chapter, the definition and properties of Ideal Lattices will not be enunciated in general terms, but they will be restricted to the cases relevant in the construction of lattice based primitives. This section follows the approach to ideal lattices adopted in \cite{ideal_lattices}, adapting it to the more restrictive setting.

\begin{definition}[Ideal Lattice]
$\mathscr{L}$ is said to be an Ideal Lattice, if it is an ideal of a ring of the form $\mathbb{Z}[x]/(f)$, where $f\in\mathbb{Z}[x]$.\\
\end{definition}

This definition already hints at the biggest advantage of ideal lattices. Indeed, the polynomial $f$ is enough to identify the lattice $\mathscr{L}=\mathbb{Z}[x]/(f)$. Then, while generic lattices require $n^2$ elements of a matrix for a complete representation, all that is required in the case of ideal lattices are the coefficients of $f$\\

\begin{definition}[Cyclic Lattice]
$\mathscr{L}$ is said to be a Cyclic Lattice, if it is an ideal of a ring of the form $\mathbb{Z}[x]/(x^n-1)$, where $f\in\mathbb{Z}[x]$ is a monic polynomial.\\
\end{definition}

It is possible to give an alternative formulation for ideal lattices, as presented in \cite{PQC}, which makes their geometry more intuitive and explains how they relate to general lattices.

% TODO alternative
\begin{definition}[Modular Convolutional Lattices]

\end{definition}
%TODO Body

Still, operations in this kind of lattices require a lot of polynomial multiplications, which are not exactly a fast operation if performed naively. Luckily, for lattices on $\mathbb{Z}_q^n$, it is possible to speed up these operations using the Number Theoretical Transform, which allows to convert polynomial multiplication in simple component-wise multiplications between arrays. This will be explored in detail is Section \ref{sec:bg:ntt}. The final cost of 

\section{Noise Distributions}
Lattice based cryptography schemes heavily rely on short random vector, typically sampled from a Discrete Gaussian distribution, used to perturb lattice elements. This section introduces a few relevant concepts on this topic.\\

\begin{definition}[Gaussian Measure]
Consider $\mathbf{x},\mathbf{c}\in\mathbb{R}^n$, and $s>0$. The Gaussian measure is defined as
\begin{equation*}
\rho_{s,\mathbf{c}}(\mathbf{x}) = exp\left[\frac{-\pi\norm{\mathbf{x}-\mathbf{c}}^2}{s^2}\right]
\end{equation*}
\end{definition}

The total measure associated to $\rho_{s,\mathbf{c}}$ can be computed as $\int_{\mathbf{x}\in\mathbb{R}^n}\rho_{s,\mathbf{c}}(\mathbf{x})d\mathbf{x}$ $=$ $s^n$. The, the continuous Gaussian distribution of parameter $s$ and centered in $\mathbf{c}$ can be defined as follows.

\begin{definition}[Gaussian Distribution]
Consider $\mathbf{x},\mathbf{c}\in\mathbb{R}^n$, and $s>0$. The continuous Gaussian distribution is defined as
\begin{equation*}
D_{s,\mathbf{c}}(\mathbf{x})=\frac{\rho_{s,\mathbf{c}}(\mathbf{x})}{s^n}
\end{equation*}
\end{definition}

This distribution would already be sufficient to sample noise efficiently, as it is easy to see that it is the sum of $n$ orthogonal one dimensional Gaussian distributions, which means that it can be approximated with arbitrary precision using known techniques. However, it is worth to spend a little effort to define a discrete counterpart, which can be more efficiently sampled when only a finite precision is available. The first step required to reach this goal is to define the Gaussian mass of a lattice $\mathscr{L}$.

\begin{definition}
Let $\mathscr{L}$ be a lattice, $\mathbf{c}\in\mathbb{R}^n$ and consider $s>0$. The Gaussian mass of $\mathscr{L}$ is defined as
\begin{equation*}
R_{s,\mathbf{c}}(\mathscr{L})=\sum_{\mathbf{y}\in\mathscr{L}}\rho_s(\mathbf{y})
\end{equation*}
where $\mathbf{y}\in\mathscr{L}$
\end{definition}

Using the concepts of Gaussian mass and Gaussian measure, the discrete Gaussian distribution $D_{\mathscr{L},s}$ can be formulated as follows.

\begin{definition}[Discrete Gaussian Distribution]
Let $\mathscr{L}$ be a lattice, $\mathbf{c}\in\mathbb{R}^n$ and consider $s>0$. The Discrete Gaussian Distribution of parameter $s$ and center $\mathbf{c}$ over $\mathscr{L}$ is the distribution induced by the measure
\begin{equation*}
\mathbb{D_{s,\mathbf{c},\mathscr{L}}}(\mathbf{x})= \frac{\rho_{s,\mathbf{c}}(\mathbf{x})}{R_{s,\mathbf{c}}(\mathscr{L}}
\end{equation*}
\end{definition}

\begin{remark}
Although the discrete Gaussian Distribution behaves differently from the continuous one, it can be proved that for a large enough $s$, they start behaving similarly. How big $s$ needs to be is determined by the smoothing parameter of a lattice.
\end{remark}

\begin{definition}[Smoothing parameter]
Let $\mathscr{L}$ be a lattice and consider $\epsilon>0$, the smoothing parameter $\eta_\epsilon(\mathscr{L})$ is defined as the smallest $s$, $s.t.$ $\rho_{s^-1}(\mathscr{L}^*\backslash\{0\})<\epsilon$
\end{definition}

Unfortunately, computing the smoothing parameter of a lattice is often impractical. However, for application purposes, having a higher bound of the smoothing parameter is enough.

\begin{theorem}
Let $\mathscr{L}$ be a lattice, and take $\epsilon=2^{-n}$. It holds
\begin{equation*}
\eta_\epsilon(\mathscr{L})\leq\frac{\sqrt{n}}{\lambda\left(\mathscr{L}\right)}
\end{equation*}
\end{theorem}

Although both tighter and more general bounds are known on the smoothing parameters, this bound is more than adequate to use when choosing the parameters of a lattice based cryptosystem. Indeed, it is the most widespread in literature, and it is used by the authors of all the cryptosystems presented in this work when justifying their parameter choice.

\section{NTT}\label{sec:bg:ntt}

The NTT is a powerful instrument in improving the efficiency of lattice based cryptography in the setting of Ideal Lattices on $\mathbb{Z}_q^n$. As seen in Section \ref{sec:bg:ideal}, one of the drawbacks of ideal lattices is the cost of polynomial multiplications. For the sake of exposition, in this section the polynomials in $\mathbb{Z}_q[x]/(f)$, where $f$ is a primitive polynomial $s.t.$ $deg(f)=n$ will be identified with the vectors of their coefficients in $\mathbb{Z}_q^n$. Using the NTT on these representations, it is possible to compute the polynomial products with a cheap component-wise product, using then the inverse transform to restore the coefficients in $\mathbb{Z}_q^n$.\\

Let's see how the NTT works. In a nutshell, it is a specialized from of the Discrete Fourier Transform that works on number fields. Indeed, the DFT works on complex vectors as follows.

\begin{definition}
Let $\mathbf{x}\in\mathbb{C}^n$, the DFT of $\mathbf{x}$ is defined as
\begin{equation*}
DFT(\mathbf{x})=\left(\sum_{m=0}^{n-1}x_me^{-ikm2\pi/n} \right)_{i=0}^{n-1}
\end{equation*}
\end{definition}

The core idea to bring this notion to number fields, is to replace the term $e^{-i2\pi/n}$, which is an $n$-th root of unity, with some integer $W_n$, which is itself an $n$-th root of the unity in the considered number field, if any.\\
For the sake of brevity and clarity, the considered number field will be $\mathbb{Z}_q^n$, with $q$ prime, so that $\mathbb{Z}_q$ is assured to have at least one $n$-th root of unity, as long as $n|(q-1)$, i.e. it is just a matter of parameter choice.

\begin{definition}[NTT]
Let $\mathbf{x}\in\mathbb{Z}_q^n$, $q$ prime, and $W_n\in\mathbb{Z}_q$ an $n$-th root of unity. Then, the NTT of $\mathbf{x}$ is defined as
\begin{equation*}
NTT(\mathbf{x})=\left(\sum_{m=0}^{n-1}x_mW_n^{km}\text{ }mod\text{ }q,\text{ }\right)_{k=0}^{n-1}  
\end{equation*}
The inverse transform, can then be computed as
\begin{equation*}
NTT^{-1}(\mathbf{x})=\left(n^{-1}\sum_{m=0}^{n-1}x_mW_n^{-km}\text{ }mod\text{ }q,\text{ }\right)_{k=0}^{n-1}
\end{equation*} 
\end{definition}

\begin{remark}
The NTT can be formulated alternatively defining the matrix $W=[W_n^{km}$ $mod$ $q]_{(k,m)\in\{0,\ldots,n-1\}\times\{0,\ldots,n-1\}}$. With the matrix $W$, the NTT of $\mathbf{x}\in\mathbb{Z}_q^n$ can be computed as $W\mathbf{x}^t$.
\end{remark}

Although the NTT does not have the same frequency interpretation of the DFT, it still has some similar properties. The most interesting property from the point of view of lattice based cryptography is the convolutional product, which behaves similarly to the one defined for the DFT.

\begin{definition}[Circular Convolution]
Let $\mathbf{x},\mathbf{y}\in\mathbb{Z}_q^n$, $q$ prime. Then, the convolution of $\mathbf{x}$ and $\mathbf{x}$ is defined as
\begin{equation*}
\mathbf{x}\ast\mathbf{y} = \left(\sum_{m=0}^{n-1}x_{m}y_{(k-m)}\right)_{k=0}^{n-1}
\end{equation*}
where $\forall i \in \{1,\ldots,n-1\} y_{-i}$ is defined as $y_{n-i-1}$. 
\end{definition}

\begin{theorem}[Convolutional product]
Let $\mathbf{x},\mathbf{y}\in\mathbb{Z}_q^n$, $q$ prime. Then
\begin{equation*}
\mathbf{x}\ast\mathbf{y} = NTT^{-1}(NTT(\mathbf{x}) \circ NTT(\mathbf{y}))
\end{equation*}
\end{theorem}
\begin{proof}
Let's consider two arbitrary $\mathbf{x},\mathbf{y}\in\mathbb{Z}_q^n$. Then
\begin{equation*}
\begin{aligned}
NTT(\mathbf{x}\ast\mathbf{y}) = NTT\left(\left(\sum_{m=0}^{n-1}x_{m}y_{(k-m)}\right)_{k=0}^{n-1}\right)\\
= \left(\sum_{i=0}^{n-1}\sum_{m=0}^{n-1}x_my_{(i-m)}W_n^{ik}\right)_{k=0}^{n-1}\\
= \left(\sum_{m=0}^{n-1}\sum_{i=0}^{n-1}x_my_{(i-m)}W_n^{ik}\right)_{k=0}^{n-1}\\
= \left(\sum_{m=0}^{n-1}x_m\sum_{i=0}^{n-1}y_{(i-m)}W_n^{(i-m)k}W_n^{mk}\right)_{k=0}^{n-1}\\
= \left(\sum_{m=0}^{n-1}x_mW_n^{mk}\sum_{i=0}^{n-1}y_{(i-m)}W_n^{(i-m)k}\right)_{k=0}^{n-1}\\
= \left(NTT(\mathbf{x})_kNTT(\mathbf{y})_k\right)_{k=0}^{n-1} = NTT(\mathbf{x})\circ NTT(\mathbf{y})
\end{aligned}
\end{equation*}
Then, applying the inverse NTT transform to both sides of the equality yields the wanted result.
\end{proof}

This is already a step in the good direction, as it is already possible to compute the product of two polynomials with the convolutional product. On the other hand, for the circular convolution to yield the product of two polynomials, the vectors of coefficients fed into it need to be padded with n zeros, thus requiring the convolution of vectors of dimension $2n$. This issue can be solved by using the negative wrapped convolution.

\begin{definition}[Negative Wrapped Convolution]
Let $\mathbf{x},\mathbf{y}\in\mathbb{Z}_q^n$, $q$ prime, and $\psi$ a $2n$-th root of unity. Then, the negative wrapped convolution of $\mathbf{x}$ and $\mathbf{x}$ is defined as
\begin{equation*}
\mathbf{x}\circledast\mathbf{y} = (\psi^{-i})_{i=0}^{n-1}\circ NTT^{-1}(NTT(\mathbf{x})\circ NTT(\mathbf{y}))
\end{equation*}
\end{definition}

Finally, with the Negative Wrapped Convolution, a polynomial product is successfully transformed into $n$-dimensional NTT transforms and component-wise multiplications.

\subsection{Hard problems in lattices}\label{sec:bg:prob}
In this section are introduced the hard problems behind the security of lattice based cryptography primitives. None of the additional structure of ideal lattices is used in the definition of this problems, making these definitions valid both for generic and ideal lattices.\\
It is worth to point out that the following problems are hard\footnote{Under the assumption of an opportune parameters choice}, unless a basis of the lattice with particular properties is known. This can already hint to the definition of cryptographic primitives based on said problems, as such a basis could effectively be used as a private key, making a generic basis for the lattice the corresponding public key.

\subsubsection{SVP}

The main formulation for the SVP is the search problem defined as follows.

\begin{definition}[Search Shortest Vector Problem (SVP)]
Let $\mathscr{L}$ be a lattice, and $B$ a basis for $\mathscr{L}$, the SVP consists in finding $\mathbf{v}\in\mathscr{L}$ $s.t.$ $\norm{\mathbf{v}} = \lambda$, where $\lambda$ is the minimum distance of the lattice $\mathscr{L}$.
\end{definition}

It is also possible to give a decisional formulation, defined as follows.

\begin{definition}[Decisional SVP]
Let $\mathscr{L}$ be a lattice, and $B$ a basis for $\mathscr{L}$, given $\mathbf{v}\in\mathscr{L}$, determine if $\norm{\mathbf{v}} = \lambda$.
\end{definition}

The SVP problem can also be generalized as follows.

\begin{definition}[Shortest Independent Vectors Problem (SIVP)]
Let $\mathscr{L}$ be a lattice, and $B$ a basis for $\mathscr{L}$, find a set $S={s_i}_{i=0}{n-1}\subset \mathscr{L}$ of linearly independent vectors $s.t$ $\forall i$ $\norm{\mathbf{s_i}} = \lambda_i$, where $\lambda_i$ is i-th successive minimum of the lattice $\mathscr{L}$.
\end{definition}

Moreover, it is worth introducing the approximate problems $SVP_\gamma$, $SIVP_\gamma$ and $GapSVP_\gamma$, which are often more useful in security reductions than the exact counterpart.

\begin{definition}[Approximate SVP]
Let $\mathscr{L}$ be a lattice, $B$ a basis for $\mathscr{L}$, and $\gamma(n)\geq1$ the approximation factor, find a $\mathbf{v}\in\mathscr{L}\backslash0$ $s.t.$ $\norm{\mathbf{v}} \leq \gamma(n)\lambda$.
\end{definition}

\begin{definition}[\text{Approximate SIVP (SIVP$_\gamma$)}]
Let $\mathscr{L}$ be a lattice, $B$ a basis for $\mathscr{L}$, and $\gamma(n)\geq1$ the approximation factor, find a set $S={s_i}_{i=0}{n-1}\subset \mathscr{L}$ of linearly independent vectors $s.t$ $\forall i$ $\norm{\mathbf{s_i}} \leq \gamma(n)\lambda_i$.
\end{definition}

\begin{definition}[\text{Approximate Decisional SVP (GapSVP$_\gamma$)}]
Let $\mathscr{L}$ be a lattice, $B$ a basis for $\mathscr{L}$, and $\gamma(n)\geq1$ the approximation factor, tell which holds between $\lambda\leq1$ or $\lambda > \gamma(n)$.
\end{definition}
\begin{remark}
An algorithm to solve the GapSVP$_\gamma$ is allowed to error if neither of the options are true.
\end{remark}

\subsubsection{CVP}
The CVP problem family is closely related to the SVP. Indeed, it can be seen as a generalization of it.

\begin{definition}[Search CVP]
Let $\mathscr{L}$ be a lattice, $B$ a basis for $\mathscr{L}$, and a vector $\mathbf{v}\in\mathscr{\mathbb{R}^n}$, find the vector $\mathbf{w}\in\mathscr{L}$ $s.t.$ $\forall\mathbf{y}\in\mathscr{L},\norm{\mathbf{v}-\mathbf{w}} < \norm{\mathbf{v}-\mathbf{y}}$.
\end{definition}

Similarly to the SVP, there are alternative formulations, such as the Decisional and Approximate SVP. All of them can be derived as they were in the SVP setting.

\subsubsection{BDD}

\begin{definition}[\text{Bounded Distance Decoding ($\alpha$-BDD)}]
Let $\mathscr{L}$ be a lattice, $B$ a basis for $\mathscr{L}$, $\alpha\in[0,1]$, and $\mathbf{t}\in\mathbb{R}^n$. Assuming that $\exists \mathbf{y}\in\mathscr{L},\norm{\mathbf{t}-\mathbf{y}} < \alpha\lambda$, find the unique lattice vector $\mathbf{v}\in\mathscr{L}$ $s.t.$ $\norm{\mathbf{t}-\mathbf{v}}<\alpha\lambda$.
\end{definition}

\begin{remark}
An algorithm to solve the $\alpha$-BDD is allowed to error if the bound on the distance is not satisfied.
\end{remark}

\subsubsection{SIS}

\begin{definition}[Short Integer Solution Problem]
Given $B\in\mathbb{Z_q^{m\times n}}$, and $\alpha > 0$, find $\mathbf{x}\in[-\alpha,\alpha]^m\backslash{0}$, $s.t.$ $B\mathbf{x}=0$ $mod$ $q$.
\end{definition}

\begin{remark}
An approximate formulation of SIS is possible and useful, while the decisional formulation is not an hard problem, so it is not to be used.
\end{remark}

\section{LWE and RLWE}
In this section, the scope of this presentation will be further narrowed, to better introduce these two problems in light of their applications. Indeed, all of the lattices considered from now on are going to be $q$-ary lattices. At the end of Section \ref{sec:bg:g_lattice}, it was mentioned that in the setting of $q$-ary lattices, $\mathbf{x}\in\mathscr{L}\Leftrightarrow\mathbf{x}$ $mod$ $q\in\mathscr{L}$. Then, it is possible to identify such lattices with their counterpart over $\mathbb{Z}_q^n$, so that will be the setting for the next two sections. This change is quite beneficial when considering the application, as having bounded coefficients for the lattice elements allows smaller representations, and saves the considerable burden of dealing with  multiprecision integer computations. \\

\subsection{The LWE problem}
The LWE problem has been one of the first candidates to provide a base for practical lattice-based cryptographic primitive. As mentioned in Section \ref{sec:bg:prob}, most of the problem introduced until now, would yield cryptographic primitives having a basis of the lattice as private key, and another as public key. This choice would be inconvenient, as a basis needs $n^2$ elements to be represented. Such a big representation would not be practical, especially for the public key, which needs to be transmitted to other parties.\\
On the other hand, the LWE shifts the focus of the problem to specific elements in the lattice, which only need $n$ elements to be represented, while the lattice itself becomes a mere parameter of the problem.\\


\begin{definition}[Search LWE]
Let $\mathscr{L}$ be a lattice, $B$ a basis for $\mathscr{L}$, $\chi[\mathbb{Z}_q^n]$ a Discrete Gaussian Distribution, and $\mathbf{s}\in\mathbb{Z}_q^n$. Given the noisy product $\mathbb{v}=Bs + e$, where $e\xleftarrow{\$}\chi[\mathbb{Z}_q^n]$, find $\mathbf{s}$.
\end{definition}

\begin{remark}
The Discrete Gaussian Distribution is chosen to have width greater than $\sqrt{n}$.
\end{remark}

\begin{theorem}
LWE $\Leftrightarrow$ BDD
\end{theorem}
\begin{proof}
Let \it{A},\it{A'} algorithms to solve, respectively LWE and BDD.
\begin{itemize}
\item LWE $\geq$ BDD. Given an instance of BDD, with bound $\alpha$, $(B,\mathbf{z}+e)$, apply \it{A} to find $\mathbf{s}$, then $B\mathbf{s}$ $mod$ $q$ $=$ $\mathbf{z}$ or $\norm{e}>\alpha$.
\item DBB $\geq$ LWE. Given an instance of LWE, $(B,B\mathbf{s}+e)$, apply \it{A'} to find $B\mathbf{s}$, then multiply on the left by $B^{-1}$ to find $\mathbf{s}$.
\end{itemize}
\end{proof}

\begin{definition}[Decision LWE]
Let $\mathscr{L}$ be a lattice, $B$ a basis for $\mathscr{L}$, $\chi[\mathbb{Z}_q^n]$ a Discrete Gaussian Distribution, and $\mathbf{s}\in\mathbb{Z}_q^n$. Given two pairs $(B, B\mathbf{s}+e)$ and $(B,b)$, where $e\xleftarrow{\$}\chi[\mathbb{Z}_q^n]$ and $\mathbf{b}\xleftarrow{\$}U[\mathbb{Z}_q^n]$, distinguish the pair generated with the noisy product from the one sampled from the uniform distribution.
\end{definition}

\begin{theorem}
Search LWE and Decisional LWE are equivalent.
\end{theorem}
\begin{proof}
It is easy to see that any algorithm solving the Search LWE would naively solve the Decisional LWE, by searching for the solution, which only exists with negligible probability if $\mathbf{b}$ is uniformly random.\\
The converse is, as always, a bit tricky. The solution to this problem is to carefully craft the instances to feed into the decisional solver to guess $\mathbf{s}$ one component at a time.
W.l.o.g., let's consider the first component.\\
First, a random $r\in\mathbb{Z}_q$ is sampled, and used to compute
\begin{equation*}
B'=B+
\begin{bmatrix}
r &&&\\
\vdots &  \multicolumn{2}{c}{\text{\large0}}\\
r &&& \\
\end{bmatrix}
\end{equation*}
then, the first component of $\mathbf{s}$ is guessed as $k\in\mathbb{Z}_q$. Finally, the guess is used to compute $\mathbf{b}'=\mathbf{b}+\left(\frac{rk}{q}\right)_{i=0}^{n-1}$. Feeding the instance $(B',b')$ in the decisional solver, it can be verified if $k$ is the correct guess for the first component of $\mathbf{s}$. This will be accomplished in at most $q$ attempts, and repeating the procedure for each component of $\mathbf{s}$, the complete vector can be recovered in at most $nq$ steps.
\end{proof}

%      TODO?           Algoritmi (fft, fast decoding)

\subsection{Drawbacks}
At the beginning of this section, it was mentioned how the LWE allows to shift the focus of the problem from the lattice itself, to its elements. Although this is a step in the good direction, a basis of the lattice is still required to carry out computations, which means that it must be somehow stored, leaving the elements required for the representation of the problem at $O(n^2)$. Even worse: if the parameters are not fixed, for instance to avoid all-for-the-price-of-one attacks, the lattice basis will have to be transmitted anyway, leaving the situation unchanged with respect to the first, simpler problems.\\
Luckily, there are some mitigations to these problems, which still allow to create practical primitives using the LWE problem, but the very existence of these issues lead to the search of a new problem that didn't have them to begin with. In the end, the most promising candidate was identified in the RLWE problem.

\section{The RLWE problem}
% TODO
%         Problema dell'RLWE [RLWE_security, RLWE_toolkit]

%                 Presentazione del problema dell'RLWE
%                 Problemi risolti dal'uso della struttura di anello:
%                 - Rappresentazione ora richiede solo O(n)
%                 - La chiave vive in Rq (1/n dello spazio)
%                 - Decoding più veloce

%         Prova di sicurezza [pqc, RLWE_security]

%                 Riduzione di SVP e SIS a RLWE e LWE 
%                 [RLWE_security per SVP -> search-RLWE -> decisional RLWE]
%                 [SVP -> BDD -> LWE]